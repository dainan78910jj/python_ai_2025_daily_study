\documentclass[12pt]{article}
\usepackage{ctex}
\usepackage{geometry}
\usepackage{graphicx}
\usepackage{amsmath}
\usepackage{float}

\geometry{a4paper, margin=1in}

\title{泰坦尼克号乘客生存预测项目总结报告}
\author{}
\date{}

\begin{document}

\maketitle

\section{项目概述}

本项目旨在分析泰坦尼克号乘客数据集,并构建一个预测模型来预测乘客的生存情况。通过对数据集的深入分析和特征工程,尝试了多种可视化方法,并构建了一个基础的机器学习模型。

\section{数据分析与处理}

\subsection{数据集检查}
训练集包含891条记录,共12列,数据类型主要包括整数、字符串和浮点数。测试集包含418条记录,共11列,缺少了生存状态这一列。

\subsection{缺失值处理}
数据集中的缺失值主要集中在以下三列:
\begin{itemize}
    \item 年龄列:共缺失177条记录
    \item 登船港列:共缺失2条记录
    \item 船舱列:共缺失687条记录
\end{itemize}

针对登船港列的缺失值,通过票号和票价信息推断,发现缺失记录的登船港大概率为S港。

对于年龄列的缺失值,采用了以下策略进行补充:
\begin{itemize}
    \item 对于与他人共享票号的乘客,如果其他同行乘客的年龄信息完整,则使用同行乘客的平均年龄进行补充
    \item 对于单独出行且年龄缺失的乘客,通过分析临近票号或临近船舱乘客的年龄信息,使用相似乘客的平均年龄进行补充
\end{itemize}

由于船舱列缺失信息过多(约占总数据的77\%),该列缺乏实际分析价值,因此未进行处理。

\subsection{特征工程}
本项目的设计方向是通过统计字母在乘客姓名中的频率来预测生存或死亡情况,因此特征工程主要集中在姓名列的处理。通过拆分姓名列,增加了两个新特征:
\begin{itemize}
    \item Title(称谓):从姓名中提取的称谓信息
    \item Freq(频率):字母在姓名中的出现频率
\end{itemize}

\section{数据可视化分析}

本项目使用了多种可视化方法对数据进行分析:
\begin{itemize}
    \item 使用柱状图分析了男性和女性的生存和死亡情况
    \item 使用散点图分析了不同年龄和姓名长度在不同船舱等级下,男性和女性的生存情况
    \item 使用热力图分析了26个字母在所有乘客姓名中的出现比例
\end{itemize}

\section{模型构建与评估}

构建了一个基本的K近邻(KNN)模型进行生存预测。模型的预测准确率为60\%多。

然而,深入分析模型输出结果发现,模型预测的未获救概率都在60\%左右,获救概率都在40\%左右。在取最大值后,模型将所有乘客都预测为未获救,因此准确率实际上等于数据集中未获救乘客的比例。

这表明所提出的算法并未在数据中发现有效的规律,模型未能学习到有意义的模式,导致训练失败。

\section{结论与展望}

本项目通过对泰坦尼克号乘客数据的分析和建模,虽然在表面上达到了60\%多的预测准确率,但实际上模型并未学习到有效的预测模式。未来可以从以下方面进行改进:

\begin{itemize}
    \item 引入更多有效的特征工程方法
    \item 尝试其他更适合的机器学习算法
    \item 进行更深入的数据探索和分析
    \item 优化模型参数和结构
\end{itemize}

\end{document}